\documentclass[UTF8,a4paper,20pt]{ctexart}
\usepackage{geometry}
\usepackage{amsmath}
\usepackage{amssymb}
\usepackage{tikz}
\usepackage{xcolor}
\usepackage{graphicx}
\usepackage{enumitem}

\geometry{left=2cm,right=2cm,top=2cm,bottom=2cm}

% 设置字体大小适合PPT
\usepackage{anyfontsize}
\renewcommand{\normalsize}{\fontsize{16pt}{24pt}\selectfont}
\renewcommand{\large}{\fontsize{20pt}{28pt}\selectfont}
\renewcommand{\Large}{\fontsize{24pt}{32pt}\selectfont}
\renewcommand{\LARGE}{\fontsize{28pt}{36pt}\selectfont}
\renewcommand{\huge}{\fontsize{32pt}{40pt}\selectfont}
\renewcommand{\Huge}{\fontsize{36pt}{44pt}\selectfont}

% 设置行间距
\linespread{1.5}

% TikZ库
\usetikzlibrary{arrows.meta,positioning,shapes,calc}

\title{\Huge\bfseries 分块算法经典例题讲解}
\author{}
\date{}

\begin{document}

\maketitle
\thispagestyle{empty}
\newpage

% ========== 问题1:第1页 - 题目陈述 ==========
\section*{\LARGE 例题1:区间乘法、区间加法与单点查询}

\subsection*{题目描述}

给出一个长度为 $n$ 的数列 $a_1, a_2, \ldots, a_n$,以及 $n$ 个操作。

\subsection*{操作类型}

\begin{itemize}[leftmargin=2cm]
    \item \textbf{操作0}(区间加法):$\mathrm{opt} = 0, l, r, c$ 
    
    将区间 $[l, r]$ 中的所有数字都加上 $c$
    
    \item \textbf{操作1}(区间乘法):$\mathrm{opt} = 1, l, r, c$
    
    将区间 $[l, r]$ 中的所有数字都乘以 $c$
    
    \item \textbf{操作2}(单点查询):$\mathrm{opt} = 2, r$
    
    查询 $a_r$ 的当前值
\end{itemize}

\subsection*{输入格式}

\begin{itemize}[leftmargin=2cm]
    \item 第一行:整数 $n$
    \item 第二行:$n$ 个整数 $a_1, a_2, \ldots, a_n$
    \item 接下来 $n$ 行:每行一个操作
\end{itemize}

\newpage

% ========== 问题1:第2页 - 解法 ==========
\subsection*{\Large 解法:分块 + 懒标记}

\subsubsection*{核心思想}

将长度为 $n$ 的数列分成 $\sqrt{n}$ 个块,每块大小约为 $\sqrt{n}$。

对每个块维护两个懒标记:
\begin{itemize}[leftmargin=2cm]
    \item $\mathrm{mul}[i]$:第 $i$ 块的乘法标记(初始为1)
    \item $\mathrm{add}[i]$:第 $i$ 块的加法标记(初始为0)
\end{itemize}

\subsubsection*{可视化示意}

\begin{center}
\begin{tikzpicture}[scale=0.9]
    % 数组元素
    \foreach \i in {0,...,8} {
        \draw (\i*1.2,0) rectangle ++(1,1);
        \node at (\i*1.2+0.5,0.5) {$a_{\i}$};
    }
    
    % 块的划分
    \draw[red, very thick] (0,-0.2) -- (0,-0.5) -- (3.6,-0.5) -- (3.6,-0.2);
    \node[red] at (1.8,-0.8) {块1};
    
    \draw[blue, very thick] (3.6,-0.2) -- (3.6,-0.5) -- (7.2,-0.5) -- (7.2,-0.2);
    \node[blue] at (5.4,-0.8) {块2};
    
    \draw[green!60!black, very thick] (7.2,-0.2) -- (7.2,-0.5) -- (10.8,-0.5) -- (10.8,-0.2);
    \node[green!60!black] at (9.0,-0.8) {块3};
    
    % 标记说明
    \node[red] at (1.8, -1.5) {$\mathrm{mul}[1], \mathrm{add}[1]$};
    \node[blue] at (5.4, -1.5) {$\mathrm{mul}[2], \mathrm{add}[2]$};
    \node[green!60!black] at (9.0, -1.5) {$\mathrm{mul}[3], \mathrm{add}[3]$};
\end{tikzpicture}
\end{center}

\subsubsection*{操作实现}

\begin{enumerate}[leftmargin=2cm]
    \item \textbf{区间加法/乘法}:
    \begin{itemize}
        \item 对于完整覆盖的块:只更新懒标记
        \item 对于部分覆盖的块:暴力修改每个元素
    \end{itemize}
    
    \item \textbf{单点查询}:
    
    查询 $a_r$ 时,返回 $a_r \times \mathrm{mul}[\mathrm{block}(r)] + \mathrm{add}[\mathrm{block}(r)]$
\end{enumerate}

\subsubsection*{标记下传规则}

当需要访问块内具体元素时:
\[
a_i \leftarrow a_i \times \mathrm{mul}[\mathrm{block}(i)] + \mathrm{add}[\mathrm{block}(i)]
\]
然后重置标记:$\mathrm{mul}[\mathrm{block}(i)] = 1, \mathrm{add}[\mathrm{block}(i)] = 0$

\newpage

% ========== 问题1:第3页 - 复杂度分析 ==========
\subsection*{\Large 复杂度分析}

\subsubsection*{空间复杂度}

\begin{itemize}[leftmargin=2cm]
    \item 原数组:$O(n)$
    \item 块标记:$O(\sqrt{n})$ 个块,每块2个标记
    \item 总空间:$O(n)$
\end{itemize}

\subsubsection*{时间复杂度}

\textbf{区间修改操作}(加法或乘法):

\begin{center}
\begin{tikzpicture}[scale=0.8]
    % 区间示意
    \foreach \i in {0,...,10} {
        \draw (\i*0.8,0) rectangle ++(0.7,0.7);
    }
    
    % 标记修改区间
    \draw[red, line width=2pt] (1.5,-0.3) -- (7.5,-0.3);
    \node[red] at (4.5,-0.7) {修改区间 $[l, r]$};
    
    % 标记不同部分
    \draw[orange, dashed, thick] (1.5,0.7) -- (1.5,1.2) -- (2.4,1.2) -- (2.4,0.7);
    \node[orange] at (1.95,1.5) {\small 散块};
    
    \draw[blue, dashed, thick] (2.4,0.7) -- (2.4,1.5) -- (7.2,1.5) -- (7.2,0.7);
    \node[blue] at (4.8,1.8) {完整块($O(\sqrt{n})$个)};
    
    \draw[orange, dashed, thick] (7.2,0.7) -- (7.2,1.2) -- (7.5,1.2) -- (7.5,0.7);
    \node[orange] at (7.8,1.5) {\small 散块};
\end{tikzpicture}
\end{center}

\begin{itemize}[leftmargin=2cm]
    \item 散块(两端不完整的块):暴力修改,$O(\sqrt{n})$
    \item 完整块:只修改标记,最多 $O(\sqrt{n})$ 个块,每个 $O(1)$
    \item \textbf{单次操作}:$O(\sqrt{n})$
\end{itemize}

\textbf{单点查询操作}:

直接通过下标计算所属块号,应用标记,$O(1)$

\textbf{总复杂度}:

$n$ 个操作,每次 $O(\sqrt{n})$ 或 $O(1)$,总时间复杂度:$\boxed{O(n\sqrt{n})}$

\newpage

% ========== 问题2:第1页 - 题目陈述 ==========
\section*{\LARGE 例题2:区间查询与区间赋值}

\subsection*{题目描述}

给出一个长度为 $n$ 的数列 $a_1, a_2, \ldots, a_n$,以及 $n$ 个操作。

\subsection*{操作内容}

每个操作由三个参数 $l, r, c$ 组成,执行以下两步:

\begin{enumerate}[leftmargin=2cm]
    \item \textbf{查询}:统计区间 $[l, r]$ 中有多少个元素等于 $c$
    \item \textbf{修改}:将区间 $[l, r]$ 中的所有元素都赋值为 $c$
\end{enumerate}

\subsection*{输入格式}

\begin{itemize}[leftmargin=2cm]
    \item 第一行:整数 $n$
    \item 第二行:$n$ 个整数 $a_1, a_2, \ldots, a_n$
    \item 接下来 $n$ 行:每行三个整数 $l, r, c$
\end{itemize}

\subsection*{示例}

初始数组:$[1, 2, 2, 3, 3]$

操作 $l=2, r=4, c=2$:
\begin{itemize}
    \item 查询结果:区间 $[2,4]$ 是 $[2, 2, 3]$,有2个元素等于2
    \item 修改后:$[1, 2, 2, 2, 3]$
\end{itemize}

\newpage

% ========== 问题2:第2页 - 解法 ==========
\subsection*{\Large 解法:分块 + 区间赋值标记}

\subsubsection*{核心思想}

将数组分成 $\sqrt{n}$ 个块,每块维护:
\begin{itemize}[leftmargin=2cm]
    \item $\mathrm{tag}[i]$:整块赋值标记(-1表示无标记)
    \item 每个元素的实际值
\end{itemize}

\subsubsection*{可视化操作流程}

\begin{center}
\begin{tikzpicture}[scale=0.85]
    % 第一行:初始状态
    \node[left] at (-0.5,2) {\textbf{初始:}};
    \foreach \i/\v in {0/1,1/2,2/2,3/3,4/3,5/1,6/4,7/4,8/5} {
        \draw (\i*1.1,1.5) rectangle ++(1,1);
        \node at (\i*1.1+0.5,2) {$\v$};
    }
    
    % 块划分
    \draw[red, thick] (0,1.3) -- (0,1.1) -- (3.3,1.1) -- (3.3,1.3);
    \draw[blue, thick] (3.3,1.3) -- (3.3,1.1) -- (6.6,1.1) -- (6.6,1.3);
    \draw[green!60!black, thick] (6.6,1.3) -- (6.6,1.1) -- (9.9,1.1) -- (9.9,1.3);
    
    % 操作说明
    \node at (5, 0.3) {操作:$l=2, r=7, c=2$(查询并赋值)};
    
    % 第二行:操作后状态
    \node[left] at (-0.5,-0.5) {\textbf{操作后:}};
    \foreach \i/\v in {0/1,1/2,2/2,3/2,4/2,5/2,6/2,7/2,8/5} {
        \draw (\i*1.1,-1) rectangle ++(1,1);
        \node at (\i*1.1+0.5,-0.5) {$\v$};
    }
    
    % 标记修改的区间
    \draw[red, line width=3pt, opacity=0.5] (2.2,-1.3) -- (8.8,-1.3);
    
    % 标记操作方式
    \node[orange] at (2.2,-1.8) {\small 散块};
    \node[blue] at (5,-1.8) {整块标记};
    \node[orange] at (7.7,-1.8) {\small 散块};
\end{tikzpicture}
\end{center}

\subsubsection*{操作实现}

\textbf{区间查询与赋值}:

\begin{enumerate}[leftmargin=2cm]
    \item \textbf{散块}(两端不完整块):
    \begin{itemize}
        \item 先下传该块的标记(如果有)
        \item 遍历散块元素,统计等于 $c$ 的个数
        \item 将散块元素逐个赋值为 $c$
    \end{itemize}
    
    \item \textbf{整块}(完全覆盖的块):
    \begin{itemize}
        \item 如果该块有标记 $\mathrm{tag}[i]$:
        \begin{itemize}
            \item 若 $\mathrm{tag}[i] = c$,贡献块大小个 $c$
            \item 否则贡献0个 $c$
        \end{itemize}
        \item 否则遍历块内所有元素统计
        \item 给整块打上标记 $\mathrm{tag}[i] = c$
    \end{itemize}
\end{enumerate}

\newpage

% ========== 问题2:第3页 - 复杂度分析 ==========
\subsection*{\Large 复杂度分析}

\subsubsection*{块大小选择}

设块大小为 $B$,则有 $\frac{n}{B}$ 个块。

\subsubsection*{单次操作时间复杂度}

\begin{center}
\begin{tikzpicture}[scale=1.2]
    % 创建表格
    \node at (0,2) {\textbf{操作部分}};
    \node at (3,2) {\textbf{时间复杂度}};
    
    \draw (-1.5,1.5) -- (5,1.5);
    
    \node[align=left] at (0,1) {散块处理};
    \node at (3,1) {$O(B)$};
    
    \node[align=left] at (0,0.3) {整块标记};
    \node at (3,0.3) {$O(\frac{n}{B})$};
    
    \node[align=left] at (0,-0.4) {整块查询(最坏)};
    \node at (3,-0.4) {$O(B \cdot \frac{n}{B}) = O(n)$};
    
    \draw (-1.5,-0.8) -- (5,-0.8);
    
    \node[align=left] at (0,-1.3) {\textbf{单次总复杂度}};
    \node at (3,-1.3) {$\boxed{O(n)}$};
\end{tikzpicture}
\end{center}

\subsubsection*{优化分析}

\textbf{关键观察}:整块查询只在块没有标记时遍历,一旦打上标记后,之后的查询都是 $O(1)$。

\textbf{均摊分析}:
\begin{itemize}[leftmargin=2cm]
    \item 每个元素最多被遍历常数次(打标记前)
    \item 打标记后的查询都是 $O(1)$
    \item 散块操作:$O(B) = O(\sqrt{n})$
    \item 整块标记:$O(\frac{n}{B}) = O(\sqrt{n})$
\end{itemize}

选择 $B = \sqrt{n}$,单次操作均摊:$O(\sqrt{n})$

\textbf{总时间复杂度}(均摊):$\boxed{O(n\sqrt{n})}$

\newpage

% ========== 问题3:第1页 - 题目陈述 ==========
\section*{\LARGE 例题3:区间开方与区间求和}

\subsection*{题目描述}

给出一个长度为 $n$ 的数列 $a_1, a_2, \ldots, a_n$,以及 $n$ 个操作。

\subsection*{操作类型}

\begin{itemize}[leftmargin=2cm]
    \item \textbf{操作0}(区间开方):$\mathrm{opt} = 0, l, r$ 
    
    对区间 $[l, r]$ 中的每个元素 $a_i$ 进行开方:
    \[
    a_i \leftarrow \lfloor \sqrt{a_i} \rfloor
    \]
    
    \item \textbf{操作1}(区间求和):$\mathrm{opt} = 1, l, r$
    
    查询区间 $[l, r]$ 中所有元素的和:
    \[
    \sum_{i=l}^{r} a_i
    \]
\end{itemize}

\subsection*{输入格式}

\begin{itemize}[leftmargin=2cm]
    \item 第一行:整数 $n$
    \item 第二行:$n$ 个整数 $a_1, a_2, \ldots, a_n$
    \item 接下来 $n$ 行:每行一个操作
\end{itemize}

\subsection*{示例}

初始:$[16, 9, 4, 1]$

操作0,$l=1, r=3$:$[4, 3, 2, 1]$

操作1,$l=1, r=4$:输出 $4+3+2+1=10$

\newpage

% ========== 问题3:第2页 - 解法 ==========
\subsection*{\Large 解法:分块 + 区间和维护}

\subsubsection*{核心思想}

\textbf{关键性质}:开方操作使数字快速减小

\begin{center}
\begin{tikzpicture}[scale=1]
    \node at (0,0) {$10^9$};
    \draw[->,thick] (0.8,0) -- (1.5,0);
    \node at (2.3,0) {$31622$};
    \draw[->,thick] (3.1,0) -- (3.8,0);
    \node at (4.4,0) {$177$};
    \draw[->,thick] (5,0) -- (5.7,0);
    \node at (6.1,0) {$13$};
    \draw[->,thick] (6.6,0) -- (7.3,0);
    \node at (7.7,0) {$3$};
    \draw[->,thick] (8.1,0) -- (8.8,0);
    \node at (9.2,0) {$1$};
    
    \node[below] at (4.5,-0.8) {约 $\log \log n$ 次后变为1};
\end{tikzpicture}
\end{center}

将数组分成 $\sqrt{n}$ 个块,每块维护:
\begin{itemize}[leftmargin=2cm]
    \item $\mathrm{sum}[i]$:第 $i$ 块的元素和
    \item $\mathrm{max}[i]$:第 $i$ 块的最大值
\end{itemize}

\subsubsection*{可视化数据结构}

\begin{center}
\begin{tikzpicture}[scale=0.8]
    % 数组
    \foreach \i/\v in {0/16,1/9,2/4,3/16,4/25,5/36} {
        \draw (\i*1.5,2) rectangle ++(1.3,1);
        \node at (\i*1.5+0.65,2.5) {$\v$};
    }
    
    % 块划分
    \draw[red, very thick] (0,1.8) -- (0,1.5) -- (4.5,1.5) -- (4.5,1.8);
    \node[red] at (2.25,1.2) {块1};
    
    \draw[blue, very thick] (4.5,1.8) -- (4.5,1.5) -- (9,1.5) -- (9,1.8);
    \node[blue] at (6.75,1.2) {块2};
    
    % 块信息
    \draw[red, thick] (0.5,0.5) rectangle ++(3,0.6);
    \node[red, align=left] at (2, 0.8) {$\mathrm{sum}[1]=29, \mathrm{max}[1]=16$};
    
    \draw[blue, thick] (5,0.5) rectangle ++(3,0.6);
    \node[blue, align=left] at (6.5, 0.8) {$\mathrm{sum}[2]=61, \mathrm{max}[2]=36$};
\end{tikzpicture}
\end{center}

\subsubsection*{操作实现}

\textbf{区间开方}:
\begin{enumerate}[leftmargin=2cm]
    \item 对于整块:如果 $\mathrm{max}[i] \leq 1$,跳过(已收敛)
    \item 否则遍历块内元素,逐个开方,更新 $\mathrm{sum}[i]$ 和 $\mathrm{max}[i]$
    \item 散块:直接暴力修改
\end{enumerate}

\textbf{区间求和}:
\begin{enumerate}[leftmargin=2cm]
    \item 整块:直接累加 $\mathrm{sum}[i]$,$O(1)$
    \item 散块:遍历累加,$O(\sqrt{n})$
\end{enumerate}

\newpage

% ========== 问题3:第3页 - 复杂度分析 ==========
\subsection*{\Large 复杂度分析}

\subsubsection*{空间复杂度}

\begin{itemize}[leftmargin=2cm]
    \item 原数组:$O(n)$
    \item 块信息:$O(\sqrt{n})$ 个块,每块存储和与最大值
    \item 总空间:$O(n)$
\end{itemize}

\subsubsection*{时间复杂度 - 关键分析}

\textbf{开方操作的收敛性}:

\begin{center}
\begin{tikzpicture}[scale=1.1]
    \draw[->] (0,0) -- (7,0) node[right] {开方次数};
    \draw[->] (0,0) -- (0,4) node[above] {数值};
    
    % 绘制曲线
    \draw[thick, blue] plot[domain=0:6, samples=100] (\x, {4*exp(-0.8*\x)});
    
    % 标记点
    \filldraw[red] (0,4) circle (2pt) node[left] {$10^9$};
    \filldraw[red] (1,1.8) circle (2pt) node[above] {$\sim 31622$};
    \filldraw[red] (2,0.8) circle (2pt);
    \filldraw[red] (3,0.4) circle (2pt);
    \filldraw[red] (5,0.1) circle (2pt) node[right] {$\leq 1$};
    
    % 标记收敛
    \draw[dashed] (0,0.2) -- (7,0.2);
    \node at (7,0.5) {收敛};
\end{tikzpicture}
\end{center}

\textbf{每个元素最多被开方 $O(\log \log V)$ 次},其中 $V$ 是元素的最大值。

\subsubsection*{操作复杂度}

\textbf{区间求和}:
\begin{itemize}[leftmargin=2cm]
    \item 整块:$O(\sqrt{n})$ 个块,每块 $O(1)$
    \item 散块:$O(\sqrt{n})$
    \item 单次:$O(\sqrt{n})$
\end{itemize}

\textbf{区间开方}(均摊):
\begin{itemize}[leftmargin=2cm]
    \item 每个元素被真正修改的次数:$O(\log \log V)$
    \item 单次操作:最坏 $O(n)$,但均摊 $O(\sqrt{n} \log \log V)$
\end{itemize}

\textbf{总时间复杂度}:$\boxed{O(n\sqrt{n} \log \log V)}$

\newpage

% ========== 问题4:第1页 - 题目陈述 ==========
\section*{\LARGE 例题4:区间生长与区间计数}

\subsection*{题目描述}

有 $n$ 株花,每株花有一个初始高度($\leq 1000$ 的自然数)。

有两个角色执行 $q$ 个操作:

\begin{itemize}[leftmargin=2cm]
    \item \textbf{Lily White}:使花儿生长
    \item \textbf{Yuka}:统计满足条件的花
\end{itemize}

\subsection*{操作类型}

\begin{itemize}[leftmargin=2cm]
    \item \textbf{操作M}(生长):\texttt{M } $l$ $r$ $h$
    
    使区间 $[l, r]$ 内所有花的高度增加 $h$
    
    \item \textbf{操作A}(询问):\texttt{A } $l$ $r$ $k$
    
    查询区间 $[l, r]$ 内有多少花的高度不低于 $k$
    (即统计满足 $a_i \geq k$ 的花的数量)
\end{itemize}

\subsection*{输入格式}

\begin{itemize}[leftmargin=2cm]
    \item 第一行:$n$(花的数量)和 $q$(操作数量)
    \item 第二行:$n$ 个整数表示初始高度
    \item 接下来 $q$ 行:每行一个操作
\end{itemize}

\subsection*{示例}

初始:$[5, 3, 8, 2]$

\texttt{M 1 3 2}:$[7, 5, 10, 2]$

\texttt{A 1 4 6}:输出2($a_1=7 \geq 6$,$a_3=10 \geq 6$)

\newpage

% ========== 问题4:第2页 - 解法 ==========
\subsection*{\Large 解法:分块 + 懒标记 + 块内排序}

\subsubsection*{核心思想}

将数组分成 $\sqrt{n}$ 个块,每块维护:
\begin{itemize}[leftmargin=2cm]
    \item $\mathrm{add}[i]$:第 $i$ 块的加法懒标记
    \item $\mathrm{sorted}[i]$:第 $i$ 块元素的有序副本
\end{itemize}

\subsubsection*{数据结构可视化}

\begin{center}
\begin{tikzpicture}[scale=0.75]
    % 原数组
    \node[left] at (-1,2.5) {\textbf{原数组:}};
    \foreach \i/\v in {0/5,1/3,2/8,3/2,4/7,5/1} {
        \draw (\i*1.5,2) rectangle ++(1.3,1);
        \node at (\i*1.5+0.65,2.5) {$\v$};
    }
    
    % 块划分
    \draw[red, very thick] (0,1.8) -- (0,1.5) -- (4.5,1.5) -- (4.5,1.8);
    \draw[blue, very thick] (4.5,1.8) -- (4.5,1.5) -- (9,1.5) -- (9,1.8);
    
    % 块信息
    \node[red, left] at (-1,1) {块1:};
    \draw[red, thick] (0,0.5) rectangle ++(4,0.8);
    \node[red] at (2, 0.9) {$\mathrm{add}[1]=0$,有序:$[3,5,8]$};
    
    \node[blue, left] at (-1,0) {块2:};
    \draw[blue, thick] (0,-0.5) rectangle ++(4,0.8);
    \node[blue] at (2, -0.1) {$\mathrm{add}[2]=0$,有序:$[1,2,7]$};
\end{tikzpicture}
\end{center}

\subsubsection*{操作实现}

\textbf{区间加法(M操作)}:
\begin{enumerate}[leftmargin=2cm]
    \item \textbf{整块}:直接增加 $\mathrm{add}[i]$ 标记,$O(1)$
    \item \textbf{散块}:
    \begin{itemize}
        \item 下传标记到块内所有元素
        \item 逐个修改散块元素
        \item 重新排序该块的有序副本,$O(\sqrt{n} \log \sqrt{n})$
    \end{itemize}
\end{enumerate}

\textbf{区间计数(A操作)}:
\begin{enumerate}[leftmargin=2cm]
    \item \textbf{整块}:
    \begin{itemize}
        \item 在有序副本中二分查找第一个 $\geq k - \mathrm{add}[i]$ 的位置
        \item 该位置右侧的元素都满足条件,$O(\log \sqrt{n})$
    \end{itemize}
    \item \textbf{散块}:遍历元素,逐个判断,$O(\sqrt{n})$
\end{enumerate}

\subsubsection*{查询可视化}

\begin{center}
\begin{tikzpicture}[scale=0.9]
    \node at (-1.5,1.5) {查询 $\geq 6$:};
    \node at (-1.5,1) {有序块:};
    
    % 有序数组
    \foreach \i/\v in {0/1,1/3,2/5,3/7,4/8,5/10} {
        \draw (\i*1.3,0.5) rectangle ++(1.2,0.8);
        \node at (\i*1.3+0.6,0.9) {$\v$};
    }
    
    % 标记 add
    \node at (4, 0) {$\mathrm{add}=2$};
    
    % 实际值
    \node at (-1.5,-0.5) {实际值:};
    \foreach \i/\v in {0/3,1/5,2/7,3/9,4/10,5/12} {
        \draw (\i*1.3,-1) rectangle ++(1.2,0.8);
        \node at (\i*1.3+0.6,-0.6) {$\v$};
    }
    
    % 二分位置
    \draw[red, thick, ->] (2.5, -1.5) -- (2.5, -1.1);
    \node[red] at (2.5, -1.8) {二分查找 $\geq 4$};
    
    % 结果
    \draw[green!60!black, thick] (2.5,-1.2) -- (2.5,-1.5) -- (7.8,-1.5) -- (7.8,-1.2);
    \node[green!60!black] at (5.5, -2.2) {4个元素满足条件};
\end{tikzpicture}
\end{center}

\newpage

% ========== 问题4:第3页 - 复杂度分析 ==========
\subsection*{\Large 复杂度分析}

\subsubsection*{空间复杂度}

\begin{itemize}[leftmargin=2cm]
    \item 原数组:$O(n)$
    \item 有序副本:每块 $O(\sqrt{n})$,共 $O(\sqrt{n})$ 个块,总计 $O(n)$
    \item 懒标记:$O(\sqrt{n})$
    \item 总空间:$O(n)$
\end{itemize}

\subsubsection*{时间复杂度}

\textbf{M操作(区间加法)}:

\begin{center}
\begin{tikzpicture}[scale=1.1]
    % 表格
    \node[align=left] at (-1.5,1.5) {\textbf{部分}};
    \node at (2,1.5) {\textbf{复杂度}};
    
    \draw (-3,1.2) -- (4,1.2);
    
    \node[align=left] at (-1.5,0.7) {整块标记};
    \node at (2,0.7) {$O(\sqrt{n})$};
    
    \node[align=left] at (-1.5,0.1) {散块修改+重排};
    \node at (2,0.1) {$O(\sqrt{n} \log \sqrt{n})$};
    
    \draw (-3,-0.2) -- (4,-0.2);
    
    \node[align=left] at (-1.5,-0.7) {\textbf{单次}};
    \node at (2,-0.7) {$\boxed{O(\sqrt{n} \log \sqrt{n})}$};
\end{tikzpicture}
\end{center}

\textbf{A操作(区间计数)}:

\begin{center}
\begin{tikzpicture}[scale=1.1]
    % 表格
    \node[align=left] at (-1.5,1.5) {\textbf{部分}};
    \node at (2,1.5) {\textbf{复杂度}};
    
    \draw (-3,1.2) -- (4,1.2);
    
    \node[align=left] at (-1.5,0.7) {整块二分};
    \node at (2,0.7) {$O(\sqrt{n} \log \sqrt{n})$};
    
    \node[align=left] at (-1.5,0.1) {散块遍历};
    \node at (2,0.1) {$O(\sqrt{n})$};
    
    \draw (-3,-0.2) -- (4,-0.2);
    
    \node[align=left] at (-1.5,-0.7) {\textbf{单次}};
    \node at (2,-0.7) {$\boxed{O(\sqrt{n} \log \sqrt{n})}$};
\end{tikzpicture}
\end{center}

\subsubsection*{总复杂度}

$q$ 个操作,每次 $O(\sqrt{n} \log \sqrt{n})$

\textbf{总时间复杂度}:$\boxed{O(q\sqrt{n} \log n)}$

\end{document}
